\documentclass[12pt]{article} % Can change 12pt to make the text larger

\usepackage{fullpage} % Makes the formatting nicer
\usepackage{extsizes} % Allows more font sizes
\usepackage{hyperref} % Allows links in the document

\title{Setup}
\author{2941 Code Tutorials} % Hacky way to get a subtitle

\begin{document}

\maketitle 
\tableofcontents

\pagebreak

\section{Introduction}

To begin your programming journey there are a few things that must be installed onto your system,
to run and edit code.
If you are using a club laptop, skip to \hyperref[sec:testing]{Testing Your Installation}.

When writing code, it is necessesary to have a program called an IDE.
An IDE, or Integrated Development Environment, is essentially a text editor (Like Microsoft Notepad on windows)
but is specifically designed for writing code.
The IDE we want to use is called Visual Studio Code, or VS Code for short.

VEX robots are programmed using a programming language called C++.
To run C++ code, we need a program called a compiler.
The compiler reads over your code,
looks for immediately obvious errors in yor code such as typos,
and converts it into an executable file that can be run directly by a computer. 
The compiler we will be using for our robot code is called PROS.

\subsection{Installation}

The first thing we want to install is VS Code.
VS Code can be downloaded from \url{https://code.visualstudio.com/}

\subsection*{Windows}

\subsection*{Linux}

\subsection*{Mac}

We want to click the big blue button that reads "Download Mac universal".
This should download a .zip file which you'll need to extract.
Then drag the new "Visual Studio Code" file into your applications folder.

\section{Testing Your Installation}
\label{sec:testing}


\end{document}